%%%%%%%%%%%%%%%%%%%%%%%%%%%%%%%%%%%%%%%%%%%%%%%%%%%%%%%%%%%%%%%%%%%%%%%%%%%%%%%%
%2345678901234567890123456789012345678901234567890123456789012345678901234567890
%        1         2         3         4         5         6         7         8

\documentclass[letterpaper, 10 pt, conference]{ieeeconf}  % Comment this line out if you need a4paper

%\documentclass[a4paper, 10pt, conference]{ieeeconf}      % Use this line for a4 paper

\IEEEoverridecommandlockouts                              % This command is only needed if 
                                                          % you want to use the \thanks command

\overrideIEEEmargins                                      % Needed to meet printer requirements.

% See the \addtolength command later in the file to balance the column lengths
% on the last page of the document

% The following packages can be found on http:\\www.ctan.org
%\usepackage{graphics} % for pdf, bitmapped graphics files
\usepackage{subfig}
\usepackage[dvipdfmx]{graphics} % for pdf, bitmapped graphics files
\usepackage{epsfig} % for postscript graphics files
%\usepackage{mathptmx} % assumes new font selection scheme installed
%\usepackage{times} % assumes new font selection scheme installed
%\usepackage{amsmath} % assumes amsmath package installed
%\usepackage{amssymb}  % assumes amsmath package installed
\usepackage{url}

\title{\LARGE \bf Classification of Familiarity Based on Cross-Correlation Features Between EEG and Music}

\author{Shintaro~Ito\,$^{1}$, Kosuke~Fukumori\,$^{1}$, Haruki~Okawa\,$^{1}$, Tomoya~Wada\,$^{1}$, and Toshihisa~Tanaka\,$^{1,2}$% <-this % stops a space
%\thanks{*This work was supported by JSPS KAKENHI Grant Number 26540102.}% <-this % stops a space
\thanks{$^{1}$Y. Kumagai, H. Okawa, T. Wada, and T. Tanaka are with Department of Electrical and Electronic Engineering at the Tokyo University of Agriculture and Technology, 2-24-16 Nakacho, Koganei-shi, Tokyo, Japan. 
$^{2}$T. Tanaka is affiliated with RIKEN Brain Science Institute, Wako-shi, Saitama, Japan (e-mail: tanakat@cc.tuat.ac.jp).
}
\thanks{
}}


\begin{document}

\maketitle
\thispagestyle{empty}
\pagestyle{empty}

%%%%%%%%%%%%%%%%%%%%%%%%%%%%%%%%%%%%%%%%%%%%%%%%%%%%%%%%%%%%%%%%%%%%%%%%%%%%%%%%
\begin{abstract}
Epilepsy is a neurological disorder which causes abnormal discharges in the brain.
Epileptic focus localization is a important factor for successful epilepsy surgery.
The intracranial electroencephalogram (iEEG) is the most used signal for detecting epileptic focus.
The iEEG signals are obtained from Juntendo University Hospital database that consists high frequency oscillations(HFOs).
HFOs is not consisted open dataset. 
In this paper, a method for the classification of focal and non-focal iEEG signals using HFOs is presented.
The iEEG signals are decomposed the into signal components of the same frequency band.
Various entropy measures calculated from  the iEEG signals.
Then, some or all of the entropies are chosen as features, which are discriminated into focal or non-focal iEEG by using the support vector machine (SVM).
Experimental results show that the proposed method is able to differentiate the focal from non-focal iEEG signals with an average classification accuracy of 
\end{abstract}


%%%%%%%%%%%%%%%%%%%%%%%%%%%%%%%%%%%%%%%%%%%%%%%%%%%%%%%%%%%%%%%%%%%%%%%%%%%%%%%%
\section{INTRODUCTION}
%1段落:てんかんについて,自動検出が求められるのは・・・
Epilepsy is one of the most common neurological disorders in the world \cite{WHO}.
Approximately 50 million people worldwide have epilepsy, which is a chronic disorder of the brain \cite{WHO}.
Epilepsy is characterized by recurrent and unpredictable seizures caused by abnormal discharges in the brain \cite{WHO}.
In surgical treatment of epilepsy, localization of seizure area is very impotant.
Because of identification of seizure area, long-term intracranial electroencepharoglam (iEEG) are recoreded from epilepsy patients.
The seizure are detected from the iEEG recoredings using visual inspection by neurological experts.
Since the inspection works needs enourmous time.
Therefore, automatic detection of epileptic focus is required.

%2段落:HFOsにかんして
Especially, it is said high-frequency oscillations (HFOs) is effective for localization of seizure area \cite{bragin2002local,matsumoto2013pathological,zijlmans2012high}.
Pathological HFOs consist of ripples (80–-250 Hz) and fastripples(FRs) (\verb|>|200,250 Hz), which are specific biomarkers of epileptogenicity \cite{iimura2017epileptogenic}.
Akiyama reported that interictalhigh-rate FRs are a potential marker of the epileptogenicity\cite{EPI:EPI3199}.

%3段落:てんかんの自動検出では,・・・がある
Many approaches for automatic detection of epileptic focus have been proposed such as classification of healthy vs. epileptic, ictal time vs. interictal time, and focal (seizure) part vs. non-focal (non-seizure) part.
The identification of time uses the nature of iEEG that has spike in seizure time and increse high frequency oscillations in seizure onset time \cite{gutierrez2001, worrell2004}.
The identification of location uses the nature of iEEG that focal iEEG signals are less random and more stationary than non-focal iEEG signals \cite{andrzejak2012}.
The most difficult type of classification is seizure focus localization from iEEG of interictal time \cite{faust2015}.

%4段落
Computer-aided solutions have been developed to assist neurologists.
Several computer-aided solutions that use signal processing and machine learning heve been proposed to localize seizure focus in epilepsy.
In this paper, a method for the classification of focal and non-focal iEEG signals using entropy is presented.
The iEEG signals were filtered into the characteristic frequency bands of the iEEG signal:
$\delta$:0.5--4 Hz; 
$\theta$:4--8 Hz; 
$\alpha$:8--13 Hz; 
$\beta$:13--30 Hz; 
$\gamma$:30--80 Hz; 
ripples:80--250 Hz; 
fastripples:250--600 Hz.
Entropy is calculated every 7 frequency bandwidth.
%The proposed method used to perform the classification of focal and non-focal iEEG signals is shown in Figure \ref{fig:method_flow}.
Various entropy measures are calculated from the iEEG signals.
Then, some or all of the entropies are selected for feature values.
Finally, these selected entropies are used for the classification of focal and non-focal iEEG classes using the support vector machine (SVM) classifier.




\section{Method}
\subsection{Dataset}
We used two datasets in this study.
One of the dataset is Bern–-Barcelona iEEG dataset \cite{andrzejak2012}.
The other is Juntendo University Hospital dataset.

\subsubsection{Bern–-Barcelona iEEG dataset}
This dataset consists of recordings of five epilepsy patients who underwent long-term intracranial iEEG recordings at the Department of Neurology of the University of Bern. 
The patients sufferd from longstanding pharmacoresistant temporal lobe epilepsy and were candidates for epilepsy surgery. 
The iEEG signals were sampled at a frequency of 512 Hz and digitally band-pass filtered between 0.5 and 150 Hz using a fourth-order Butterworth filter. 
The multichannel iEEG recordings were labeled focal signals or non-focal signals. 
Focal signals were recorded at epileptogenic zone. 
Non-focal signals were recorded at brain area where were not involved in seizure onset. 
The dataset consists of 3750 focal iEEG signals pairs and 3750 non-focal iEEG signals pairs. Each signal contains a pair of two iEEG signals which are recorded from adjacent channels. 
The length of iEEG signals in the dataset is 20 seconds. 
The iEEG signals during seizure activity and three hours after the last seizure activity were excluded. 

\subsubsection{Juntendo University Hospital dataset}
This  iEEG signals from four patients who had temporal lobe epilepsy of focal cortical dysplasia. The iEEG signals are recorded at the Juntendo University. 
Multichannel iEEG signals were recorded with intracranial electrodes. 
The iEEG signals were sampled at 2 kHz. 
The multichannel iEEG are labeled focal and non-focal channels by experts. 
Experts labeled focal channels that detected first ictal EEG signal changes All other channels included in the recordings were labeled as non-focal channels. 
In this paper, we analyze interictal (non seizure time) iEEG signals of 2 hours.

\subsection{Entropy-Based Features Extraction}
Brief descriptions of these entropies is given in this section.

\subsubsection{Spectral Entropy}
Spectral entropy quantifies the complexity of the signal to indicate regularity of time series.
It uses the amplitude components of the power spectrum of the signal as the probabilities in entropy calculations. 
In this paper, we have used Shannon entropy (ShEn), Renyi entropy (RenEn) and Tsallis' entropy (TsEn)
These entropies are obtained by multiplying the power in each frequency $p_f$.
These entorpies can be defined as \cite{kannathal2005,renyi1961measures,chen2014tsallis}:
\begin{equation}
ShEn = \sum_f p_f \log  \left(1/p_f\right)
\end{equation}
\begin{equation}
RenEn(\alpha) = -\frac{1}{1-\alpha}\log  \sum_f p_f^\alpha 
\end{equation}
\begin{equation}
TsEn(q) = -\frac{1}{1-q}\biggl( 1 - \sum_f p_f^q \biggl)
\end{equation}

\subsubsection{Phase Entropy}
Phase entropy is an indicator of complexity of time series, defined through a bispectrum known as higher order spectra \cite{nikias1993}.
Phase entropy is calculated from the bispectrum similar to spectral entropy.
The bispectrum of the signal can be defined as:
\begin{equation}
B(f_1, f_2) = X(f_1)X(f_2)X^*(f_1 + f_2)
\end{equation}
where $X(f)$ is the Fourier transform of a signal $x(i)$.
Two type of phase entropy, S1 and S2, are defined as:
\begin{equation}
S1 = - \sum _k p_k \log (p_k), \:\:\:\: 
S2 = -\sum_j q_j \log (q_j)
\end{equation}
where
\begin{equation}
p_k = \frac{|B(f_1, f_2)|}{\sum_{\Omega \in f_1, f_2} |B(f_1, f_2)|}, \:\:\:\:
q_j = \frac{|B(f_1, f_2)|^2}{\sum_{ \Omega \in f_1, f_2 } |B(f_1, f_2)|}
\end{equation}

\subsubsection{Approximate Entropy and Sample Entropy}
Approximate entropy (ApEn) and sample entropy (SpEn) are an indicator of complexity of time siries \cite{pincus1991}.
These entropies detects changes of the signal variation and compares the similarity of the samples by pattern length and similarity coefficient.
SpEn is a modified version of the ApEn \cite{richman2000}.
ApEn is difined as:
\begin{eqnarray}
ApEn(d, r, N) = &\frac{1}{(N-d+1)} \sum_{i-1}^{N-d+1} \ln \left( C_i^d (r) \right) \nonumber\\
&- \frac{1}{(N-d)} \sum_{i-1}^{N-d} \ln \left( C_i^{d+1} (r) \right)
\end{eqnarray}
where the $d$ is the embedding dimension, $r$ is a fixed parameter which sets the tolerance of the comparison and $C_i^d (r)$ is the correlation integral.
$C_i^d (r)$ is difined as :
\begin{equation}
C_i^d(r) = \frac{1}{(N-d+1)} \sum_{j-1}^{N-d+1} \Theta \left(r - ||X(i)-X(j)|| \right)
\end{equation}
where $X(i)$ and $X(j)$ are subsequence of time series $x(i)$ and $\Theta$ is the Heaviside function. 
Also, SpEn is difined as: 
\begin{equation}
SpEn(d, r, N) = -\ln \left( \frac{A^d(r)}{B^d(r)} \right)
\end{equation}
where 
\begin{equation}
B^d(r) = \frac{1}{(N-d)} \sum_{1}^{N-d} C_i^d (r), \:\:\:\
\end{equation}
\begin{equation}
A^d(r) = \frac{1}{(N-d)} \sum_{1}^{N-d} C_i^{d+1} (r)
\end{equation}

\subsubsection{Permutation Entropy}


\subsection{Features Selection}
Mutual information, a measure of linear and non-linear interdependence between features, is also used as a feature selection criterion\cite{orosco2013survey}.
Feature selection evaluates mutual information between individual features and class labels\cite{estevez2009normalized}.
In this paper, Mutual information is used to serect entropy.

\begin{equation}
I(A;B) = \sum_{a\in A}\sum_{b\in B}P_{A,B}(a,b)\log\frac{P_{A,B}(a,b)}{P_A(a)P_B(b)}
\end{equation}


\subsection{Classification}
Some of the calculated seven types of the entropies are chosen as features for SVM classifier.
Combination of entropy is shown in Tables \ref{tab:result_ave_ari} and \ref{tab:result_ave_nashi}.
Number of IMFs are selected based on statistically significant features ($p \leq 0.05$) on Student's $t$-test.
The selected features are input to SVM classifier with radial basis function kernel.
SVM classifier distinguished to focal and non-focal iEEG signals.


\section{RESULTS AND DISCUSSION}
\begin{table*}[t]
	\begin{center}
		\caption{ Mutual infomation in Bern–Barcelona iEEG dataset}
		\label{tab:bern_entropy_mutual_info}
		\scalebox{1}{
			\begin{tabular}{|c|c|c|c|c|c|c|c|}
				\hline
				\multicolumn{1}{|c|}{{}} & \multicolumn{6}{c|}{Frequancy Band}  \\ 
				\hline
				entropy & δ & θ & α & β & γ & ripple \\
				\hline
				Permutation	&	0.01001877	&	0.08369443	&	0.02786098	&	0.02074077	&	0.02054239	&	0.0287999	\\
				Approximate	&	0.04043817	&	0.00583419	&	0.01410552	&	0.00543831	&	0	&	0.03026465	\\
				Sample	&	0.01489264	&	0	&	0.02159091	&	0.05448126	&	0.05377349	&	0.02800181	\\
				Phase(S1)	&	0.00656706	&	0	&	0.01117957	&	0.00603118	&	0.0340779	&	0.01303872	\\
				Phase(S2)	&	0.01130014	&	0	&	0.00056818	&	0.01421173	&	0.01745291	&	0.00934858	\\
				Tsallis'	&	0	&	0.02261457	&	0	&	0.02663556	&	0.01635527	&	0.00810457	\\
				Shannon	&	0	&	0.00519527	&	0	&	0.02347162	&	0.0290724	&	0.01374545	\\
				Renyi	&	0	&	0.02301328	&	0	&	0.0265711	&	0.0129557	&	0.00589675	\\
				\hline
			\end{tabular}
		}
	\end{center}
\end{table*}
Table \ref{tab:bern_entropy_mutual_info} shows mutual infomation of each entropy in Bern–Barcelona iEEG dataset.


\begin{table}[t]
	\begin{center}
		\caption[Sum of mutual infomation of each entropy in Bern–Barcelona iEEG dataset ]{Sum of mutual infomation of each entropy in Bern–Barcelona iEEG dataset}
		\label{tab:bern_entropy_mutual_info_sum}
		\begin{tabular}{|c|c|c|}
			\hline
			entropy & Sum of mutual infomation & Rank \\ \hline
			Permutation	&	0.19165724	&	1	\\
			Sample	&	0.17274011	&	2	\\
			Approximate	&	0.09608084	&	3	\\
			Tsallis'	&	0.07370997	&	4	\\
			Shannon	&	0.07148474	&	5	\\
			Phase(S1)	&	0.07089443	&	6	\\
			Renyi	&	0.06843683	&	7	\\
			Phase(S2)	&	0.05288154	&	8	\\
			\hline
		\end{tabular}
	\end{center}
\end{table}



\section{CONCLUSION}
\label{sec:conclusion}
% Conclusion



%\addtolength{\textheight}{-12cm}   % This command serves to balance the column lengths
                                  % on the last page of the document manually. It shortens
                                  % the textheight of the last page by a suitable amount.
                                  % This command does not take effect until the next page
                                  % so it should come on the page before the last. Make
                                  % sure that you do not shorten the textheight too much.

%%%%%%%%%%%%%%%%%%%%%%%%%%%%%%%%%%%%%%%%%%%%%%%%%%%%%%%%%%%%%%%%%%%%%%%%%%%%%%%%
%%%%%%%%%%%%%%%%%%%%%%%%%%%%%%%%%%%%%%%%%%%%%%%%%%%%%%%%%%%%%%%%%%%%%%%%%%%%%%%%
%%%%%%%%%%%%%%%%%%%%%%%%%%%%%%%%%%%%%%%%%%%%%%%%%%%%%%%%%%%%%%%%%%%%%%%%%%%%%%%%
%\section*{ACKNOWLEDGMENT}
%This work was supported by JSPS Grant 16K12456 and the Daiwa Anglo-Japanese Foundation.

\bibliographystyle{IEEEtran}
\bibliography{ito}

\end{document}
